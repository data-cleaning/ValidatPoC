\title{Layout of the PoC}
In order to do a proof of concept on a common set of validation cases in two
different countries it was necessaru to agree on the process being executed.
For this we made a process diagram, which is shown in figure \textbf{TODO}.

The validation rules to be used in the PoC were selected from two sources:

 - The repository of about 1800 validation rules used in practice in different
   countries. These rules were taken from the survey that was returned by
\textbf{TODO} countries. Note that these validation rules were free text, so they were
actually expressed in a variety of formats (SQL, natural languags, SAS, Blaise,
etc.). We selected.
 
 - The rules mentioned in the "Study on VTL". We have added rules from this
   study because they express an interesting variety of possibilities in terms
of validation.
 
This implied a set of 18 validation rules of varying complexity.

It would have been optimal if the PoC could use real data to work on. For the
rules from the survey data could be available at the NSI's, but for the rules
originating from the study this was not possible. For practical reasons, we
decided to generate synthetic data sets for all 18 rules.  As the ESSnet noted
that missing data is a fact of life in statistics, missing fields were added in
the synthetic data sets as well.  To facilitate checking the correctness of the
validation to be done, a "expected" field was added to the the data sets which
has values "valid", "invalid" or "undecided".  Both validation rules and
synthetic datasets are available on github
(https://github.com/data-cleaning/ValidatPoC).

The 18 test validation rules were translated into VTL 1.0 by a VTL expert.  At
the side of Germany as well as in the Netherlands a neutral person, not
involved with the ESSnet but with sufficient knowledge of programming languages
and the national validation language, was assigned to the task of translating
the VTL rules into the national dialect.  This person was asked not to have a
look at the original version of the rule, to better imitate the use of VTL as a
communication language between international and national organisations.

This resulted in implementations of the rules in the eSattistik and validate
languages. These implementations were used to validate the data sets. Both the
validation results as well as the three different ways of expressing the rules
(VTL, eStatistik, validate) were compared.

