\title{Implementations}

\vspace{0.6 cm}

\noindent
This chapter presents the various implementations of the data validaion rules described in \ref{implementations}.

\section{VTL}

VTL

\section{Validate}

The \code{validate} package \citep{loo:2015} is build on top of the popular and
R language for statistical computing \citep{rcore:2015}. It is
implemented as an `R package' -- a strictly standardized way of distributing R
software, and publicly available through the Comprehensive R Archive Network
(CRAN). 

\subsubsection{The R language and environment}
The R language is an open source project, supported by the R foundation seated
in Vienna, Austria\footnote{\code{https://www.r-project.org/foundation/}}.
Over the last decade R received a surge in popularity both from research and
data science and business analytics communities. As a result R currently
integrates with every popular software system for data storage and processing,
including Microsoft SQL server, Oracle databases, Tibco Spotfire, Spark, and
Hadoop to name but a few.  Interaction with other programming environments such
as \code{C} and \code{C++}, \code{Java}, \code{.Net}, and \code{javascript} are
readily and freely available as well. The recently established R
consortium\footnote{\code{https://www.r-consortium.org/}} is an industry
initiative aimed to streamline and fund further developments in R.


The validate package is intended to be a small and powerful package that can be
used standalone by an analyst using R, or integrated easily into bigger
projects using one of the facilities mentioned above. As such, it is build with
strong adherence to the famous Unix
philosophy\footnote{\code{https://en.wikipedia.org/wiki/Unix\_philosophy}}: do
one thing, and do it really well.


\subsubsection{Approach for validate}
The philosophy behind the validate package is to reuse existing solutions as
much as possible. Most importantly, this means that rather than defining a
language from scratch, the validate package utilizes a subset of existing R
syntax for defining data validation rules. As an example, consider the following rule from the ESSnet survey \citep{walsdorfer:2015}.
\begin{quote}
Check whether the relative occurrence of the category high in a column
containing values low, high, medium does not exceed 10%.
\end{quote}
We assume a simple test dataset, with at least a column called \code{level}.
\begin{center}
\begin{tabular}{|c|}
\hline
\textbf{level}\\
\hline
medium\\
\hline
low\\
\hline
medium\\
\hline
medium\\
\hline
high\\
\hline
$\vdots$\\
\hline
\end{tabular}
\end{center}
In validate, the above rule translates to the following syntax.
\begin{verbatim}
  counts :=  table(level)
  counts["high"] < 0.1 * sum(counts)
\end{verbatim}
Here, \code{table} is a cross tabulation function that tabulates occurrences
of values in the \code{level} column. Since this function is a standard R
feature, tabulation or counting is not part of validate's syntax. Indeed the
validate package only makes sure that the statements issued by a user are in
fact validating statements as defined in the Hanbook on validation developed
for this project \citep{zio:2015}. In particular, this means that every
(statistical) transformation available to R can be part of a validating
statements.  With thousands of functions in base R, combined with the more than
7600 user-contributed packages available on CRAN, this results in a truly rich
set of validation functionality with a minimal cost of development and
maintenance to the authors.

Finally, we note that the approach of reusing R syntax for a specific purpose
is not new. Indeed several authors have created popular packages using the same
technology. Examples include an implementation of the famous grammar of
graphics \citep{wilkinson:2006} by \citet{wickham:2014}, the  `formula
interface' for statistical model specification and a grammar for data
manipulation by \cite{wickham:2014}. In fact, reusing existing languages, in
particular functional languages like R, is a well-established technique in
computer science.  Some key references for developing such embedded domain
specific languages (DSL) are \cite{fowler:2011} and \cite{gibbons:2015}.


\subsubsection{Features and usage}
Given that statistical functionality comes for free with R, the validate
package focuses only on features directly related to execution and maintenance
of validation rules and  analyses and presentation of validation results.
A few of the features are highlighted below.

\begin{description}
\item[Rule definition.] Rules can be defined on the R commandline for interactive
use or in text files that can be stored and manipulated as source code. Rule sets
may also be stored in a data base, since R by default supports connectivity to 
(nearly) any database that is (commercially) available.

\item[Rule metadata.] If so desired, rules can be provided with metadata such as
a time stamp, short and long descriptions, and the origin of the rule.  By
default, validate uses the widely supported YAML format for metadata
\citep{ben:2009}, mainly beacause of it's human-readability. Since YAML can be
translated to formats like XML or JSON at the push of a button, communication
of rules between different softwares or machines is very easy.

\item[Rule organisation.] Dependencies between text files defining rules are
supported to allow for easy reuse of rules across different datasets.  A type
of \code{include} statement allows for hyrachical dependencies between rule
sets.

\item[Rule execution.] Validation rules can be confronted with any data that can
be read into R, which encompasses every common data format currently in use.
Of particular interest is \code{SDMX}, which is supported for R through a
package of \cite{bondel:2015}.

\item[Rule analyses.] Automatic detection of rule types, variable occurrence and
the rule dependency graph are currently built in. Add-on packages are currently
prototyped that will allow for detection of redundancies or contradictory rule
sets.

\item[Extensibility.] This is borrowed from R's extensibility. Users may define
functions which can be used in validating statements. The package needs no special
provisions for that.

\item[Integration.] Integration and communication with other is borrowed from
the strong features that R already has in this area. Because of its openness,
many interfaces with existing software and platforms, including commercial
ones, have been developed.

\item[Analyses of results.] The package has basic functionality to aggregate and
visualize results of validation  procedures. For example, one can sort records
according to the number of rule violations (record prioritization) or rules
according to which have been violated most often. Such summaries, and visualisations
thereof are important tools for monitoring data quality and the efficacy of 
data cleaning process steps \citep{pannekoek:2014}.
\end{description}

The \code{validate} package is the successor of an earlier package of the same
authors named \code{editrules}. The latter package has been used in production
for several years at Statistics Netherlands and is being evaluated by several
other national statistical institutes. At the moment, validate is being
introduced as editrules' replacement. The validate package is backwards
compatible with editrules in that it can read the same input files. However,
validate extends the earlier package by allowing for a much wider range of
validation rules (editrules was restricted to certain types of in-record
validation rules that are suitable for an error-localization algorithm).




\section{eSTATISTIK}

eSTATISTIK

