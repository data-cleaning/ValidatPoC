\textbf{Note regarding eSTATISTIK}: for improving accessibility to non-German speaking readers, rules appear in an English version. The translation was limited to keywords, variable names and comments. The original version can be found on [GIT]. Also, in cases where the logic of a rule was split over two or more files, the files and their respective programmatic type (procedure or rule) are given. For an explanation of why such splitting may happen, please see chapter~\ref{implementations}.
\linebreak
\linebreak
\textbf{General note}: code snippets may be re-formatted to better fit the page boundaries.

\subsubsection*{  Rule 1: Natural language}
\begin{quote}

Number of hours per week usually worked should be between 1 and 80

Missing data results in undecided.


\end{quote}
\subsubsection*{The VTL implementation}
\begin{verbatim}
  DS= person-id, hours_worked

  DSr:= DS#hours_worked between 1 and 80
  /* In case a value in hours_worked is NULL the value returned will be NULL */
  .
\end{verbatim}
\subsubsection*{The eStatistik implementation}

\code{Rule\_01.txt} contains the rule for validating the column  \code{hours\_worked} and is configured as a hard check (this configuration takes place in the rule editor). \code{Rule\_01.undecided.txt} only checks if \code{hours\_worked} is empty and represents a soft check. The procedure \code{Rule\_01\_control.txt} combines these checks, invoking \code{Rule\_01.txt} only if \code{Rule\_01.undecided.txt} returns FALSE, in which case \code{hours\_worked} is not empty. In this way, a soft error will be registered and displayed if the value is missing, and a hard error if a value is present but not within the valid range.
\linebreak
\linebreak
Procedure \code{Rule\_01\_control.txt}:

\begin{verbatim}
  IF CHECK Undecided = FALSE  
    THEN CHECK Rule_01
  END
\end{verbatim}

\noindent
Rule \code{Rule\_01.txt}:

\begin{verbatim}
  NOT hours_worked IN SEQUENCE (1 ++ 80)
\end{verbatim}

\noindent
Rule \code{Rule\_01\_undecided.txt}:

\begin{verbatim}
  hours_worked = EMPTY
\end{verbatim}

\subsubsection*{The validate implementation.}
\begin{verbatim}
  # rule_01:
  hours_worked >= 1 & hours_worked <= 80
\end{verbatim}


\newpage

\subsubsection*{  Rule 2: Natural language}
\begin{quote}


cost + profit = turnover

Missing data results in undecided



\end{quote}

\subsubsection*{The VTL implementation}
\begin{verbatim}
  DS= business-id, cost, profit, turnover

  DSr:= (DS#cost + DS#profit) = DS#turnover
\end{verbatim}

\subsubsection*{The eStatistik implementation}
\noindent
Procedure \code{Rule\_02\_control.txt}:
\begin{verbatim}
  IF cost /= EMPTY AND profit /= EMPTY 
    THEN CHECK Rule_02
  END
\end{verbatim}
Rule \code{Rule\_02.txt}:
\begin{verbatim}
  turnover /= cost + profit
\end{verbatim}

\subsubsection*{The validate implementation.}
\begin{verbatim}
  # rule_02:
  cost + profit == turnover
\end{verbatim}


\newpage

\subsubsection*{  Rule 3: Natural language}
\begin{quote}


Check whether the relative occurrence of the category \code{high} in a column containing values \code{low}, \code{high}, \code{medium} does not exceed 10\%.

Missing values are ignored when determining the relative occurrences.


\end{quote}

\subsubsection*{The VTL implementation}
\begin{verbatim}
  DS=level

  DScalc:= DS[calc 1 as "temp_id" role "identifier"", 1 as "msrcount" role
"measure"]
  DSr:= DScalc[filter level="high"][aggregate count(msrcount)]<=(( DScalc
[aggregate count (msrcount)])*0.1)
\end{verbatim}

\subsubsection*{The eStatistik implementation}
\noindent
Rule \code{Rule\_03.txt}:
\begin{verbatim}
  DECLARE rc, s1  , z1 , total
  rc,z1,total := {0,0,0}

  FOR EVERY s1  IN DATASET mat_Rule03 (level1)

    IF s1 /= 'NA' THEN  total := total + 1 END "count records that contain any valid value "
     IF s1  = 'high' THEN z1 := z1 + 1   END    "count records that contain value "high"

  END

  "Check relative occurrence of value 'high'"

  IF z1 > total * 0.1 THEN rc := 1 END

  RETURN rc
\end{verbatim}

\subsubsection*{The validate implementation.}
\begin{verbatim}
  # def_03:
  counts := table(level)

  # rule_03:
  counts["high"] < 0.1 * sum(counts)
\end{verbatim}


\newpage

\subsubsection*{  Rule 4: Natural language}
\begin{quote}


Price change between the current month and the previous month should not exceed 50\% (taking the previous value as 100\%). The same must hold for the price change between the current month and the same month last year.

Missing values result in invalid



\end{quote}
\subsubsection*{The VTL implementation}
\begin{verbatim}
  DS=id, item, price_t, price_t-1, price_Y-1

  DSr1:= ((DS#price_t - DS#price_t-1) <= (DS#price_t-1 * 0.5)) and ((DS#price_t +
DS#price_Y-1) <= (DS#price_Y-1 * 0.5))

  /* if a NULL value is in one of the terms of the AND the result will be as
indicated in 3VL, Three-valued logic see page 42 VTL-part1 */
\end{verbatim}
\subsubsection*{The eStatistik implementation}
\begin{verbatim}
  (price_t = EMPTY OR price_t_1 = EMPTY OR price_Y_1 = EMPTY)
  OR
  FUNCTION ABSOLUTEVALUE (price_t - price_t_1)  > price_t_1 * 0.5
  OR
  FUNCTION ABSOLUTEVALUE (price_t - price_Y_1)  > price_Y_1 * 0.5\end{verbatim}
\subsubsection*{The validate implementation.}
\begin{verbatim}
  # rule_04:
  (price_t - price_tmin1) <= 0.5 * price_tmin1 & (price_t - price_Ymin1) <=
  0.5 * price_Ymin1
\end{verbatim}


\newpage

\subsubsection*{  Rule 5: Natural language}
\begin{quote}


Age of grandparents $–$ $28$ $\leq$ age of their grandchildren

Missing values result in undecided. Results are reported per grandchild.



\end{quote}
\subsubsection*{The VTL implementation}
\begin{verbatim}
  DS= id(identifier), age, grandchild_of

  DSmerge:=merge(DS as "DSgp",DS as "DSgc"
  on (DSgp#person-id= DSgc# grandchild_of),
  return (DSgc#person-id as "person-id", DSgc#age as "age"", DSgp#age as "gp_age",
DSgc#grandchild_of as "grandchild_of")

  DSr:= (DSmerge#gp_age-28) >= DSmerge#age

  DSinvalid:=DS setdiff DSr[keep(person-id,age,grandchild_of)]

\end{verbatim}
\subsubsection*{The eStatistik implementation}
\noindent
Procedure \code{Rule\_05\_control.txt}:
\begin{verbatim}
  IF grandchild_of /= EMPTY AND age /= EMPTY
    THEN CHECK Rule_05
  END
\end{verbatim}
\noindent
Rule \code{Rule\_05.txt}:
\begin{verbatim}
  DECLARE rc, tmp_age
  tmp_age := EMPTY

  tmp_age := DATASET mat_Rule05lb (person_id = grandchild_of ; age)

  IF tmp_age - 28 < age
    THEN rc := 1
  END

  RETURN rc
\end{verbatim}

\subsubsection*{The validate implementation.}
\begin{verbatim}
  # def_age_gp:
  age_gp := age[match(grandchild_of, person_id)]

  # rule_04:
  age_gp - 28 >= age
\end{verbatim}


\newpage

\subsubsection*{  Rule 6: Natural language}
\begin{quote}


If a product is out of season, the price and quantity must be the same as last month's values.

Missing values result in undecided.


\end{quote}
\subsubsection*{The VTL implementation}
\begin{verbatim}
  DS= product_id, season, price_t, quantity_t, price_t-1, quantity_t-1

  DSout:=DS[filter season="out"]
  DSr:= (DSout#price_t= DSout#price_t-1) and (DSout#quantity_t=
DSout#quantity_t-1)

  /* if a NULL value is in one of the terms of the AND the result will be as
indicated in 3VL, Three-valued logic
  see page 42 VTL-part1 */
\end{verbatim}

\subsubsection*{The eStatistik implementation}
\begin{verbatim}
  season = 'out' UND (price_t /= price_t_1 ODER quantity_t /= quantity_t_1)
\end{verbatim}

\subsubsection*{The validate implementation.}
\noindent
Procedure \code{Rule\_06\_control.txt}:
\begin{verbatim}
  IF price_t /= EMPTY AND price_t_1 /= EMPTY  AND  quantity_t /= EMPTY   AND quantity_t_1  /= EMPTY 
    THEN CHECK Rule_06
  END
\end{verbatim}
\noindent
Rule \code{Rule\_06.txt}:
\begin{verbatim}
  season = 'out' AND (price_t /= price_t_1 OR quantity_t /= quantity_t_1)
  \end{verbatim}


\newpage

\subsubsection*{  Rule 7: Natural language}
\begin{quote}


The price change of a single item may not influence the change in the mean prices by more than 10\\%, upwards or downwards.

Explanation in detail. Let \code{m(t)} denote the mean price at time \code{t} and \code{m(t-1)} the mean time at time \code{t-1}.
The change in mean prices is given by \code{d(t,t-1) = abs(m(t) - m(t-1))}. Also define \code{m(t,i)}, which is the
mean price at time \code{t}, but the price of the \code{i}th item is set equal to the price at time \code{t-1}. Accordingly, we write \code{d(t,t-1,i) = abs(m(t,i)-m(t-1))}. The rule now states that
\begin{verbatim}
0.9 <= d(t,t-1,i)/d(t,t-1) <= 1.1
\end{verbatim}

We assume all data is available.


\end{quote}
\subsubsection*{The VTL implementation}
\begin{verbatim}
  DS=product-id(identifier),price_t , price_tm1
  DScalc:= DS[calc 1 as "temp_id" role "identifier"][keep (temp_id, price_t ,
price_mt1)]
  DSmt:= DScalc [keep (temp_id,price_t)][aggregate avg(price_t)]
  DSmt_1:= DScalc [keep (temp_id,price_mt1)][aggregate avg(price_mt1)]
  DScount:=DS[keep (temp_id,price_t)][aggregate count(price_t)]
  DSr:=(abs(DSmt - DSmt_1 + (DScalc#price_mt1- DScalc#price_t)/DScount))/abs(DSmt-
DSmt_1)) between 0.9 and 1.1
\end{verbatim}
\subsubsection*{The eStatistik implementation}
\begin{verbatim}
  DECLARE rc, d_t, d_tm1, s_t, s_tm1, d_t_neu, t, tm1, counter,  DSr
  rc, d_t, d_tm1, s_t, s_tm1, d_t_neu, counter  := {0,0,0,0,0,0,0}

"Count totals SP2 and SP3 across all records"
  FOR EVERY  t, tm1 IN DATASET mat_Rule07lb (price_t, price_tm1)
          counter := counter + 1
          s_t   := s_t + t
          s_tm1 := s_tm1 + tm1

  END

  "Evaluate result"

  IF counter > 0

    THEN 
      "Compute previous average"
      d_t       := s_t   / counter
      d_tm1     := s_tm1 / counter
	
      "Compute new average"
      d_t_neu := (s_t - price_t + price_tm1) / counter

      "Compute relative size of new average"
      DSr :=  FUNCTION ABSOLUTEVALUE(d_t - d_tm1) / FUNCTION ABSOLUTEVALUE(d_t_neu - d_tm1)

      "Check"
      IF NOT DSr IN SEQUENCE  (0.9 ++ 1.1)
        THEN rc := 1
      END

  END 
 
  RETURN rc
\end{verbatim}
\subsubsection*{The validate implementation.}
\begin{verbatim}
  # def_ratio:
  ratio := abs(mean(price_t) - mean(price_tm1) + (price_tm1 -
  price_t)/length(price_t)/(mean(price_t) - mean(price_tm1)))

  # rule_07:
  ratio >= 0.9 && ratio <= 1.1
\end{verbatim}


\newpage

\subsubsection*{  Rule 8: Natural language}
\begin{quote}


Year of birth in household questionnaire must equal year of birth in individual questionnaire

Missing values result in undecided.

Results are reported per person.


\end{quote}
\subsubsection*{The VTL implementation}
\begin{verbatim}
  DS_h= household-id, person-id(identifier),person, year_of_birth
  DS_p= person-id(identifier),person, year_of_birth, gender

  DSr:= DS_h#year_of_birth=DS_p#year_of_birth
\end{verbatim}
\subsubsection*{The eStatistik implementation}
\noindent
Procedure \code{Rule\_08\_control.txt}:
\begin{verbatim}
  IF year_of_birth /= EMPTY
    THEN CHECK Rule_08
  END
\end{verbatim}
\noindent
Rule \code{Rule\_08.txt}:
\begin{verbatim}
  DECLARE rc
  rc := 0

  "Haushalt = household"

  IF NOT  SEQUENCE (person_id , person , year_of_birth) 
    IN DATASET Haushalt (person_id , person , year_of_birth)
    THEN rc := 1
  END

  RETURN rc
\end{verbatim}

\subsubsection*{The validate implementation.}
\begin{verbatim}
  # rule_08:
  person_id == person$person_id
\end{verbatim}


\newpage

\subsubsection*{  Rule 9: Natural language}
\begin{quote}


The \code{forall} quantifyer signifies that the rule is satisfied for a data set when \emph{all} records satisfy the rule.

\begin{verbatim}
forall x: x.age >= 0 AND x.age <= 113
\end{verbatim}

Missing values result in undecided.


\end{quote}
\subsubsection*{The VTL implementation}
\begin{verbatim}
  DS=person-id(identifier), age

  DScalc:= DS[calc 1 as "temp_id" role "identifier"][keep (temp_id, age)]

  DScond:= DScalc[filter age between 0 and 113]

  DSr:=DScond[aggregate count(age)]= DScalc[aggregate count(include NULLS age)]

\end{verbatim}
\subsubsection*{The eStatistik implementation}
\begin{verbatim}
  DECLARE tmp_age, rc, tmp_decided, tmp_invalid
    rc,tmp_decided, tmp_invalid := {0,0,0}
  tmp_age := EMPTY
  FOR EVERY tmp_age IN DATASET mat_Rule09 (age)

    IF tmp_age  = EMPTY 
      THEN
        tmp_decided := 1 
    ELSE
      IF NOT tmp_age  IN SEQUENCE (0++113)
        THEN tmp_invalid := 1  
      END
    END
  END

  IF tmp_decided = 0 AND  tmp_invalid = 1
    THEN rc := 1
  END

  RETURN rc
\end{verbatim}
\subsubsection*{The validate implementation.}
\begin{verbatim}
  # rule_09:
  all(age >= 0 && age <= 113)
\end{verbatim}


\newpage

\subsubsection*{  Rule 10: Natural language}
\begin{quote}


\begin{verbatim}
exists x: x.business-id = 100 AND x.turnover > 1.000.000
\end{verbatim}

We assume all data is available.


\end{quote}
\subsubsection*{The VTL implementation}
\begin{verbatim}
  DScalc:= DS[calc 1 as "temp_id" role "identifier "]
  DScond:= DScalc[filter business_id=100 and turnover>1000000]
  DSr:= DScond[keep (temp_id,business_id)][aggregate count (business_id)] > 0
\end{verbatim}
\subsubsection*{The eStatistik implementation}
\begin{verbatim}
  DECLARE rc, tmp_turnover 
  rc := 1
  tmp_turnover := EMPTY

  FOR EVERY tmp_turnover IN DATASET mat_Rule10 (business_id = '100' ; turnover )


    IF tmp_turnover /= EMPTY AND tmp_turnover > 1000000
      THEN rc := 0
    END

  END

  RETURN rc
\end{verbatim}
\subsubsection*{The validate implementation.}
\begin{verbatim}
  # rule_10:
  any(business_id == 100 & turnover > 1e+06)
\end{verbatim}


\newpage

\subsubsection*{  Rule 11: Natural language}
\begin{quote}


The \code{exists!} quantifier signifies 'there exists exactly one'.


\begin{verbatim}
exists! x: x.business-id = 100 AND x.turnover > 1.000.000
\end{verbatim}

A missing value resuls in undecided, except when the truth value can be determined regardless of the missing value.



\end{quote}
\subsubsection*{The VTL implementation}
\begin{verbatim}
  DScalc:= DS[calc 1 as "temp_id" role "identifier"]
  DScond:= DScalc[filter business_id=100 and turnover>1000000]
  DSr:= DScond[keep (temp_id,business_id)][aggregate count (business_id)] = 1
\end{verbatim}
\subsubsection*{The eStatistik implementation}
\begin{verbatim}
  DECLARE rc, tmp_turnover, tmp_undecided, tmp_count
  rc,tmp_undecided, tmp_count := {1,0,0}
  tmp_turnover := EMPTY

  FOR EVERY tmp_turnover IN DATASET mat_Rule11 (business_id = '100' ; turnover )

  IF tmp_turnover  = EMPTY 
    THEN
      tmp_undecided := 1 
    ELSE
      IF tmp_turnover > 1000000
        THEN tmp_count :=  tmp_count + 1  
      END
    END
  END

   IF tmp_undecided = 1 OR  tmp_count = 1
    THEN rc := 0
  END
 
  RETURN rc
\end{verbatim}
\subsubsection*{The validate implementation.}
\begin{verbatim}
  # rule_11:
  sum(business_id == 100 & turnover > 1e+06) == 1
\end{verbatim}


\newpage

\subsubsection*{  Rule 12: Natural language}
\begin{quote}


\begin{verbatim}
forall x: 
  IF x.relation_to_head = 4 
  THEN exists y:
    x.spouse-id = y.person-id AND y.relation_to_head = 3
\end{verbatim}

We assume all data is available.


\end{quote}
\subsubsection*{The VTL implementation}
\begin{verbatim}
  DS=person-id(identifier), spouse-id, relation_to_head

  DSfilter := DS[filter relation_to_head = 4]
  DSmerge := merge(DS "DSx",DS "DSy",
  on
  (DSy#spouse-id = DSx#person-id and DSy#relation_to_head = 3 and
DSx#relation_to_head = 4)
  return
  (DSx#person-id as "person-id"))

  DSnot_exists := DSfilter not_exists_in DSmerge

  DScount := DSnot_exists[calc 1 as "id" role "identifier"][keep (id, person_id)]
[aggregate count (person_id)] = 0
\end{verbatim}
\subsubsection*{The eStatistik implementation}
\begin{verbatim}
  DECLARE rc,tmp_relation_to_head
  rc := {0}
  tmp_relation_to_head := EMPTY

  IF relation_to_head ='4'
  THEN   

    IF NOT spouse_id IN DATASET mat_Rule12 (person_id)
      THEN
        rc := 1
      ELSE
        tmp_relation_to_head :=  DATASET mat_Rule12 (person_id = spouse_id ; relation_to_head )

        IF tmp_relation_to_head /= '3'
          THEN  rc := 1
        END
    END

  END
  
  RETURN rc
\end{verbatim}
\subsubsection*{The validate implementation.}
\begin{verbatim}
  # def_rel_4:
  rel_4 := person_id[relation_to_head == 4]

  # def_rel_3:
  spouse_of_rel_3 := spouse[relation_to_head == 3]

  # rule_12:
  all(rel_4 %in% spouse_of_rel_3)
\end{verbatim}


\newpage

\subsubsection*{  Rule 13: Natural language}
\begin{quote}


The combination of sex and age group in the data set is unique, i.e., there do not exist two distinct records in
the data set with an identical combination of values for sex and age group.

- sex groups: \code{male}, \code{female}
- age groups: \code{child}, \code{adult}, \code{senior} 

We assume all data is available.


\end{quote}
\subsubsection*{The VTL implementation}
\begin{verbatim}
  DS= person-id(identifier),gender(identifier),age-group(identifier)
  /*
  * gender: male, female
  * age-groups: child, adult, senior
  */
  DScalc := DS[calc 1 as "id" role "identifier", 1 as "msrcount" role "measure"]
  DScount := DS[keep(id, msrcount, gender, age_groups)][aggregate count(msrcount)]
[filter msrcount > 1]
  DSr := DScount [keep (id, msrcount)][aggregate count(msrcount)] = 0
\end{verbatim}
\subsubsection*{The eStatistik implementation}
\begin{verbatim}
  DECLARE rc, dummy, counter
  rc, counter  := {0,0}

  FOR EVERY dummy IN DATASET mat_Rule13 (gender = gender, age_group = age_group  ; person_id )
    counter := counter + 1
    IF counter /= 1
     THEN rc := 1
    END

  END
  RETURN rc
\end{verbatim}
\subsubsection*{The validate implementation.}
\begin{verbatim}
  # def_counts:
  counts := table(gender, age_group)

  # rule_13:
  counts <= 1
\end{verbatim}


\newpage

\subsubsection*{  Rule 14: Natural language}
\begin{quote}


Every combination of sex and age group occurs at least once in the data set.

- sex groups: \code{male}, \code{female}
- age groups: \code{child}, \code{adult}, \code{senior} 

We assume all data is available.


\end{quote}
\subsubsection*{The VTL implementation}
\begin{verbatim}
  DS=person-id(identifier), name, gender(identifier), age-group(identifier)
  DSgender= gender(identifier) {male, female}
  DSage =age-group(identifier) {child, adult, senior}
  /*
  * gender: male, female
  * age-groups: child, adult, senior
  */
  DSmerge := merge(DSgender "DSgender" ,DSage "DSage" ,
  on
  (1 = 1)
  return
  (DSgender#gender as "gender",DSage #age-group as "age-group"))
  DSdiff := DSmerge setdiff DS[keep (gender, age-group)]
  DSr := DSdiff [calc 1 as "msrcount" role "measure"][aggregate count(msrcount)] =
0
\end{verbatim}
\subsubsection*{The eStatistik implementation}
\begin{verbatim}
  DECLARE rc, tmp_age_group, tmp_gender, male_child, female_child, male_adult, female_adult, male_senior, female_senior
  rc,male_child, female_child,male_adult,female_adult,male_senior,female_senior  := {0,0,0,0,0,0,0}

  FOR EVERY tmp_gender , tmp_age_group IN DATASET mat_Rule14 (gender , age_group)
    IF tmp_gender = 'male'   AND tmp_age_group = 'child'
      THEN male_child    := male_child    + 1
    END
    IF tmp_gender = 'female' AND tmp_age_group = 'child'   
      THEN female_child  := female_child  + 1
    END
    IF tmp_gender = 'male'   AND tmp_age_group = 'adult'  
      THEN male_adult    := male_adult    + 1
    END
    IF tmp_gender = 'female' AND tmp_age_group = 'adult'  
      THEN female_adult  := female_adult  + 1
    END
    IF tmp_gender = 'male'   AND tmp_age_group = 'senior' 
      THEN male_senior   := male_senior   + 1
    END
    IF tmp_gender = 'female' AND tmp_age_group = 'senior' 
      THEN female_senior := female_senior + 1
    END

  END
  
  IF male_child = 0 OR female_child = 0 OR male_adult = 0 OR female_adult = 0 OR male_senior = 0 OR female_senior = 0 
    THEN rc := 1
  END

  RETURN rc
\end{verbatim}
\subsubsection*{The validate implementation.}
\begin{verbatim}
  # def_counts:
  counts := table(gender, age_group)

  # rule_gender:
  gender %in% c("male", "female")

  # rule_ag:
  age_group %in% c("child", "adult", "senior")

  # rule_complete:
  counts == 1 && sum(counts) == 6
\end{verbatim}


\newpage

\subsubsection*{  Rule 15: Natural language}
\begin{quote}


If two records have the same postal code, they must have the same value for \code{city}. Below, this is expressed
as a functional dependency (https://en.wikipedia.org/wiki/Functional\_dependency)

\begin{verbatim}
postal_code --> city
\end{verbatim}



\end{quote}
\subsubsection*{The VTL implementation}
\begin{verbatim}
  DS=district-id(identifier),postcode(identifier),city(identifier)


  DScount := DS[calc 1 as msr_count role "MEASURE"]
  DSr := DScount[keep (postcode, msr_count)][aggregate count (msr_count)] =
  DScount[keep (postcode, city, msr_count)][aggregate count (msr_count)][keep
(postcode, msr_count)]
\end{verbatim}
\subsubsection*{The eStatistik implementation}
\noindent
Procedure \code{Rule\_15\_control.txt}:
\begin{verbatim}
  IF postcode /= EMPTY
    THEN CHECK Rule_15
  END
\end{verbatim}
\noindent
Rule \code{Rule\_15.txt}:
\begin{verbatim}
  DECLARE rc, tmp_city, counter
  rc, counter  := {0,0}

  FOR EVERY tmp_city IN DATASET mat_Rule15 (postcode = postcode ; city )
    IF city /= tmp_city
      THEN rc := 1
    END
  END
  
  RETURN rc
\end{verbatim}

\subsubsection*{The validate implementation.}
\begin{verbatim}
  # rule_15:
  postcode ~ city
\end{verbatim}


\newpage

\subsubsection*{  Rule 16: Natural language}
\begin{quote}


The following is a check on hierarchical aggreggation.

\begin{verbatim}
forall k >= 1: w(x1. ... .xk) equals the sum of
w(x1. ... .xk.i) forall i >= 0
\end{verbatim}

We assume all data is available.



\end{quote}
\subsubsection*{The VTL implementation}
\begin{verbatim}
  DS=id(identifier),level(identifier),weight

  /*
  * Create a hierarchy (actually is no possible to do using VTL because some
string operators are missing)
  *
  * MAPS FROM MAPS TO LEVEL SIGN
  * x1 1 +
  * x1.1 x1 2 +
  * x1.2 x1 2 +
  * x1.3 x1 2 +
  * x2 1 +
  * x2.1 x2 2 +
  */

  DShierarchy := hierarchy(DS, level, "HRC", false)
  DScond := (DShierarchy = DS)[filter weight = "false"]
  DSr := DScond[calc 1 as "msrcount" role "MEASURE"][aggregate count(msrcount)] =
0
\end{verbatim}
\subsubsection*{The eStatistik implementation}
\begin{verbatim}
  DECLARE rc, tmp_AnzSst, tmp_such, tmp_level, tmp_sum, tmp_weight, hit
  rc,tmp_sum,hit  := {0,0,0}

  tmp_AnzSst := FUNCTION LENGTH (level)

  IF tmp_AnzSst IN SEQUENCE  (1,3)
    THEN 
      FOR EVERY tmp_level, tmp_weight IN DATASET mat_Rule16 (level, weight )

        IF tmp_AnzSst = 1 AND FUNCTION LENGTH (tmp_level) = 3 AND FUNCTION PART (tmp_level,1,1) = FUNCTION PART (level,1,1) 
          THEN
            tmp_sum:= tmp_sum + tmp_weight  
            hit := 1
        END

        IF tmp_AnzSst = 3 AND FUNCTION LENGTH (tmp_level) = 5 AND FUNCTION PART (tmp_level,1,3) = FUNCTION PART (level,1,3) 
          THEN
            tmp_sum:= tmp_sum + tmp_weight  
            hit := 1
        END

    END

    "Check"
    IF tmp_sum /= weight AND hit = 1
      THEN rc := 1
    END

  END

  RETURN rc
\end{verbatim}
\subsubsection*{The validate implementation.}
\begin{verbatim}
  # def_parent:
  with_parent := sub("\\.?\\d+$", "", level)

  # def_is_parent:
  is_parent := level %in% with_parent

  # def_par_lev:
  parent_levels := level[is_parent]

  # def_childsum:
  childsum := tapply(weight, with_parent, sum)

  # rule_16:
  weight[is_parent] == childsum[parent_levels]
\end{verbatim}


\newpage

\subsubsection*{  Rule 17: Natural language}
\begin{quote}


The value for no\_of\_household\_members must equal the number of records for each household


\end{quote}
\subsubsection*{The VTL implementation}
\begin{verbatim}
  DShousehold=household-id(identifier),members
  DSpersons=person-id(identifier),household-id(identifier) (in the example fields
are not correctly defined)

  DScount := (DSpersons[calc 1 as "members" role "MEASURE"][keep (household-id,
members)][aggregate count(members)]=
  DShousehold)[filter members= "false"]
  DSr := DScount[calc 1 as "msr_count" role "MEASURE"][aggregate count(msr_count)]
= 0

\end{verbatim}
\subsubsection*{The eStatistik implementation}
\noindent
Procedure \code{Rule\_17\_control.txt}:
\begin{verbatim}
  IF members /= EMPTY
    THEN CHECK Rule_17
  END
\end{verbatim}
\noindent
Rule \code{Rule\_17.txt}:
\begin{verbatim}
  DECLARE rc, tmp_dummy, counter
  rc, counter  := {0,0}

  FOR EVERY tmp_dummy IN DATASET personen (household_id = household_id ; person_id )
    counter := counter + 1
  END

  IF counter /= members
    THEN rc := 1
  END

  RETURN rc
\end{verbatim}

\newpage

\subsubsection*{  Rule 18: Natural language}
\begin{quote}


This last one is a bit complicated. It involves two files, one with households (\code{x}) and one with persons data (\code{y}). In the household file, it is registered how many members there are, say 3. It is then expected that
there are persons with \code{person-id} 1,2,3 in the file \code{y}. The rule is satisfied if for all households, all person-id's can be found, and the id's have the correct values.


\begin{verbatim}
forall x: forall n:
  IF 1 <= n <= x.no\emph{of}household\_members
  THEN exists y: 
    x.household-id = y.household-id AND y.person-id = n
\end{verbatim}


We assume all data is available.

\end{quote}
\subsubsection*{The VTL implementation}
\begin{verbatim}
  DShousehold=household-id(identifier),members
  DSpersons=household-id(identifier), person-id(identifier)


  DSmerge:=merge (DShousehold as "DSh", DSpersons as "DSp"
  on DSh#household-id=DSp#household-id,
  return
  (DSh#household-id as household-id,DSh#person-id as person-id,DSp#members as
members))


  DSout:= DSmerge[filter person-id < 1 or person-id>members][keep (household-
id,members)][aggregate count (members)] = 0

  DSdist:= DSmerge[rename (person-id) as "p_id" role "measure"][aggregate
count_distinct (p_id)][filter p id <> members]
  [aggregate count (members)] = 0


  DSr := (DSout and DSdist)

\end{verbatim}

\subsubsection*{The eStatistik implementation}
\noindent
Procedure \code{Rule\_18\_control.txt}:
\begin{verbatim}
  IF members /= EMPTY
    THEN CHECK Rule_18
  END
\end{verbatim}
\noindent
Rule \code{Rule\_18.txt}:
\begin{verbatim}
  DECLARE rc, tmp_person_id, x1
  rc := 0

  LOOP FOR x1 := 1 UNTIL x1 > members
    IF NOT SEQUENCE (household_id, x1) IN DATASET personen (household_id , person_id )
      THEN rc := 1
    END
  END

  RETURN rc
\end{verbatim}

\subsubsection*{The validate implementation.}
\begin{verbatim}
  # def_max:
  total_members := household$members[match(household_id, household$household_id)]

  # rule18:
  person_id >= 1 && person_id <= total_members
\end{verbatim}

