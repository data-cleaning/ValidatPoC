\textbf{Note regarding eSTATISTIK}: for improving accessibility to non-German readers, rules appear in an English version. The translation was limited to keywords, variable names and comments. The original version can be found on [GIT]. Also, in cases where rules where split over two or more files, corresponding remarks were made. for an explanation of why this may happen, please see chapter~\ref{implementations}.

\subsubsection*{  Rule 1: Natural language}
\begin{quote}


Number of hours per week usually worked should be between 1 and 80

Missing data results in undecided.


\end{quote}
\subsubsection*{The VTL implementation}
\begin{verbatim}
  DS= person-id, hours_worked

  DSr:= DS#hours_worked between 1 and 80
  /* In case a value in hours_worked is NULL the value returned will be NULL */
  .
\end{verbatim}
\subsubsection*{The eStatistik implementation}

Procedure \code{Rule\_01\_control.txt}:

\begin{verbatim}
  IF CHECK Undecided = FALSE  
    THEN CHECK Rule_01
  END
\end{verbatim}

\noindent
Rule \code{Rule\_01.txt}:

\begin{verbatim}
  NOT hours_worked IN SEQUENCE (1 ++ 80)
\end{verbatim}

\noindent
Rule \code{Rule\_01\_undecided.txt}:

\begin{verbatim}
  hours_worked = EMPTY
\end{verbatim}

\subsubsection*{The validate implementation.}
\begin{verbatim}
  # rule_01:
  hours_worked >= 1 & hours_worked <= 80
\end{verbatim}


\newpage

\subsubsection*{  Rule 2: Natural language}
\begin{quote}


cost + profit = turnover

Missing data results in undecided



\end{quote}
\subsubsection*{The VTL implementation}
\begin{verbatim}
  DS= business-id, cost, profit, turnover

  DSr:= (DS#cost + DS#profit) = DS#turnover
\end{verbatim}
\subsubsection*{The eStatistik implementation}
\begin{verbatim}
  turnover /= cost + profit
\end{verbatim}
\subsubsection*{The validate implementation.}
\begin{verbatim}
  # rule_02:
  cost + profit == turnover
\end{verbatim}


\newpage

\subsubsection*{  Rule 3: Natural language}
\begin{quote}


Check whether the relative occurrence of the category \code{high} in a column containing values \code{low}, \code{high}, \code{medium} does not exceed 10\%.

Missing values are ignored when determining the relative occurrences.


\end{quote}
\subsubsection*{The VTL implementation}
\begin{verbatim}
  DS=level

  DScalc:= DS[calc 1 as "temp_id" role "identifier"", 1 as "msrcount" role
"measure"]
  DSr:= DScalc[filter level="high"][aggregate count(msrcount)]<=(( DScalc
[aggregate count (msrcount)])*0.1)
\end{verbatim}
\subsubsection*{The eStatistik implementation}
\begin{verbatim}
  VAR rueck, s1 , z1 , total
  rueck,z1,total := {0,0,0}

  FUER JEDES s1 IN MATERIAL mat_Rule03 (level1)

  WENN s1 /= 'NA' DANN total := total + 1 ENDE "Gesamtsätze ohne LEER Werte (NA)"
  WENN s1 = 'high' DANN z1 := z1 + 1 ENDE "Anz. high "

  ENDE

  "Auswertung"
  WENN z1 > total * 0.1 DANN rueck := 1 ENDE

  RUECKGABE rueck
\end{verbatim}
\subsubsection*{The validate implementation.}
\begin{verbatim}
  # def_03:
  counts := table(level)

  # rule_03:
  counts["high"] < 0.1 * sum(counts)
\end{verbatim}


\newpage

\subsubsection*{  Rule 4: Natural language}
\begin{quote}


Price change between the current month and the previous month should not exceed 50\% (taking the previous value as 100\%). The same must hold for the price change between the current month and the same month last year.

Missing values result in invalid



\end{quote}
\subsubsection*{The VTL implementation}
\begin{verbatim}
  DS=id, item, price_t, price_t-1, price_Y-1

  DSr1:= ((DS#price_t - DS#price_t-1) <= (DS#price_t-1 * 0.5)) and ((DS#price_t +
DS#price_Y-1) <= (DS#price_Y-1 * 0.5))

  /* if a NULL value is in one of the terms of the AND the result will be as
indicated in 3VL, Three-valued logic see page 42 VTL-part1 */
\end{verbatim}
\subsubsection*{The eStatistik implementation}
\begin{verbatim}
  (price_t = LEER ODER price_t_1 = LEER ODER price_Y_1 = LEER)
  ODER
  FUNKTION ABSOLUTWERT (price_t - price_t_1) > price_t_1 * 0.5
  ODER
  FUNKTION ABSOLUTWERT (price_t - price_Y_1) > price_Y_1 * 0.5
\end{verbatim}
\subsubsection*{The validate implementation.}
\begin{verbatim}
  # rule_04:
  (price_t - price_tmin1) <= 0.5 * price_tmin1 & (price_t - price_Ymin1) <=
  0.5 * price_Ymin1
\end{verbatim}


\newpage

\subsubsection*{  Rule 5: Natural language}
\begin{quote}


Age of grandparents $–$ $28$ $\leq$ age of their grandchildren

Missing values result in undecided. Results are reported per grandchild.



\end{quote}
\subsubsection*{The VTL implementation}
\begin{verbatim}
  DS= id(identifier), age, grandchild_of

  DSmerge:=merge(DS as "DSgp",DS as "DSgc"
  on (DSgp#person-id= DSgc# grandchild_of),
  return (DSgc#person-id as "person-id", DSgc#age as "age"", DSgp#age as "gp_age",
DSgc#grandchild_of as "grandchild_of")

  DSr:= (DSmerge#gp_age-28) >= DSmerge#age

  DSinvalid:=DS setdiff DSr[keep(person-id,age,grandchild_of)]

\end{verbatim}
\subsubsection*{The eStatistik implementation}
\begin{verbatim}
  VAR rueck, hf_age
  hf_age := LEER

  hf_age := MATERIAL mat_Rule05lb (person_id = grandchild_of ; age)

  WENN hf_age - 28 < age
  DANN rueck := 1
  ENDE


  RUECKGABE rueck
\end{verbatim}
\subsubsection*{The validate implementation.}
\begin{verbatim}
  # def_age_gp:
  age_gp := age[match(grandchild_of, person_id)]

  # rule_04:
  age_gp - 28 >= age
\end{verbatim}


\newpage

\subsubsection*{  Rule 6: Natural language}
\begin{quote}


If a product is out of season, the price and quantity must be the same as last month's values.

Missing values result in undecided.


\end{quote}
\subsubsection*{The VTL implementation}
\begin{verbatim}
  DS= product_id, season, price_t, quantity_t, price_t-1, quantity_t-1

  DSout:=DS[filter season="out"]
  DSr:= (DSout#price_t= DSout#price_t-1) and (DSout#quantity_t=
DSout#quantity_t-1)

  /* if a NULL value is in one of the terms of the AND the result will be as
indicated in 3VL, Three-valued logic
  see page 42 VTL-part1 */
\end{verbatim}
\subsubsection*{The eStatistik implementation}
\begin{verbatim}
  season = 'out' UND (price_t /= price_t_1 ODER quantity_t /= quantity_t_1)
\end{verbatim}
\subsubsection*{The validate implementation.}
\begin{verbatim}
  # rule_06:
  if (season == "out") price_t == price_tmin1 && quantity_t ==
  quantity_tmin1
\end{verbatim}


\newpage

\subsubsection*{  Rule 7: Natural language}
\begin{quote}


The price change of a single item may not influence the change in the mean prices by more than 10\\%, upwards or downwards.

Explanation in detail. Let \code{m(t)} denote the mean price at time \code{t} and \code{m(t-1)} the mean time at time \code{t-1}.
The change in mean prices is given by \code{d(t,t-1) = abs(m(t) - m(t-1))}. Also define \code{m(t,i)}, which is the
mean price at time \code{t}, but the price of the \code{i}th item is set equal to the price at time \code{t-1}. Accordingly, we write \code{d(t,t-1,i) = abs(m(t,i)-m(t-1))}. The rule now states that
\begin{verbatim}
0.9 <= d(t,t-1,i)/d(t,t-1) <= 1.1
\end{verbatim}

We assume all data is available.


\end{quote}
\subsubsection*{The VTL implementation}
\begin{verbatim}
  DS=product-id(identifier),price_t , price_tm1
  DScalc:= DS[calc 1 as "temp_id" role "identifier"][keep (temp_id, price_t ,
price_mt1)]
  DSmt:= DScalc [keep (temp_id,price_t)][aggregate avg(price_t)]
  DSmt_1:= DScalc [keep (temp_id,price_mt1)][aggregate avg(price_mt1)]
  DScount:=DS[keep (temp_id,price_t)][aggregate count(price_t)]
  DSr:=(abs(DSmt - DSmt_1 + (DScalc#price_mt1- DScalc#price_t)/DScount))/abs(DSmt-
DSmt_1)) between 0.9 and 1.1
\end{verbatim}
\subsubsection*{The eStatistik implementation}
\begin{verbatim}
  VAR rueck, d_t, d_tm1, s_t, s_tm1, d_t_neu, t, tm1, counter, DSr
  rueck, d_t, d_tm1, s_t, s_tm1, d_t_neu, counter := {0,0,0,0,0,0,0}

  "Summenbildung SP2 und SP3 über alle Sätze "
  FUER JEDES t, tm1 IN MATERIAL mat_Rule07lb (price_t, price_tm1)
  counter := counter + 1
  s_t := s_t + t
  s_tm1 := s_tm1 + tm1

  ENDE

  "Berechnung:"
  WENN counter > 0

  DANN
  "Durchschnittberechnung"
  d_t := s_t / counter
  d_tm1 := s_tm1 / counter
  "Neuer Durchschnittswert"
  d_t_neu := (s_t - price_t + price_tm1) / counter

  "Formel"
  DSr := FUNKTION ABSOLUTWERT(d_t - d_tm1) / FUNKTION ABSOLUTWERT(d_t_neu - d_tm1)


  WENN NICHT DSr IN REIHE (0.9 ++ 1.1)
  DANN rueck := 1
  ENDE

  ENDE

  RUECKGABE rueck
\end{verbatim}
\subsubsection*{The validate implementation.}
\begin{verbatim}
  # def_ratio:
  ratio := abs(mean(price_t) - mean(price_tm1) + (price_tm1 -
  price_t)/length(price_t)/(mean(price_t) - mean(price_tm1)))

  # rule_07:
  ratio >= 0.9 && ratio <= 1.1
\end{verbatim}


\newpage

\subsubsection*{  Rule 8: Natural language}
\begin{quote}


Year of birth in household questionnaire must equal year of birth in individual questionnaire

Missing values result in undecided.

Results are reported per person.


\end{quote}
\subsubsection*{The VTL implementation}
\begin{verbatim}
  DS_h= household-id, person-id(identifier),person, year_of_birth
  DS_p= person-id(identifier),person, year_of_birth, gender

  DSr:= DS_h#year_of_birth=DS_p#year_of_birth
\end{verbatim}
\subsubsection*{The eStatistik implementation}
\begin{verbatim}
  VAR rueck
  rueck := 0

  WENN NICHT REIHE (person_id , person , year_of_birth)
  IN MATERIAL Haushalt (person_id , person , year_of_birth)
  DANN rueck := 1
  ENDE

  RUECKGABE rueck
\end{verbatim}
\subsubsection*{The validate implementation.}
\begin{verbatim}
  # rule_08:
  person_id == person$person_id
\end{verbatim}


\newpage

\subsubsection*{  Rule 9: Natural language}
\begin{quote}


The \code{forall} quantifyer signifies that the rule is satisfied for a data set when \emph{all} records satisfy the rule.

\begin{verbatim}
forall x: x.age >= 0 AND x.age <= 113
\end{verbatim}

Missing values result in undecided.


\end{quote}
\subsubsection*{The VTL implementation}
\begin{verbatim}
  DS=person-id(identifier), age

  DScalc:= DS[calc 1 as "temp_id" role "identifier"][keep (temp_id, age)]

  DScond:= DScalc[filter age between 0 and 113]

  DSr:=DScond[aggregate count(age)]= DScalc[aggregate count(include NULLS age)]

\end{verbatim}
\subsubsection*{The eStatistik implementation}
\begin{verbatim}
  VAR hf_age, rueck, hf_decided, hf_invalid
  rueck,hf_decided, hf_invalid := {0,0,0}
  hf_age := LEER
  FUER JEDES hf_age IN MATERIAL mat_Rule09 (age)

  WENN hf_age = LEER
  DANN hf_decided := 1

  SONST
  WENN NICHT hf_age IN REIHE (0++113)
  DANN hf_invalid := 1
  ENDE
  ENDE
  ENDE

  WENN hf_decided = 0 UND hf_invalid = 1
  DANN rueck := 1
  ENDE

  RUECKGABE rueck
\end{verbatim}
\subsubsection*{The validate implementation.}
\begin{verbatim}
  # rule_09:
  all(age >= 0 && age <= 113)
\end{verbatim}


\newpage

\subsubsection*{  Rule 10: Natural language}
\begin{quote}


\begin{verbatim}
exists x: x.business-id = 100 AND x.turnover > 1.000.000
\end{verbatim}

We assume all data is available.


\end{quote}
\subsubsection*{The VTL implementation}
\begin{verbatim}
  DScalc:= DS[calc 1 as "temp_id" role "identifier "]
  DScond:= DScalc[filter business_id=100 and turnover>1000000]
  DSr:= DScond[keep (temp_id,business_id)][aggregate count (business_id)] > 0
\end{verbatim}
\subsubsection*{The eStatistik implementation}
\begin{verbatim}
  VAR rueck, hf_turnover
  rueck := 1
  hf_turnover := LEER

  FUER JEDES hf_turnover IN MATERIAL mat_Rule10 (business_id = '100' ; turnover )


  WENN hf_turnover /= LEER UND hf_turnover > 1000000
  DANN rueck := 0
  ENDE

  ENDE

  RUECKGABE rueck
\end{verbatim}
\subsubsection*{The validate implementation.}
\begin{verbatim}
  # rule_10:
  any(business_id == 100 & turnover > 1e+06)
\end{verbatim}


\newpage

\subsubsection*{  Rule 11: Natural language}
\begin{quote}


The \code{exists!} quantifier signifies 'there exists exactly one'.


\begin{verbatim}
exists! x: x.business-id = 100 AND x.turnover > 1.000.000
\end{verbatim}

A missing value resuls in undecided, except when the truth value can be determined regardless of the missing value.



\end{quote}
\subsubsection*{The VTL implementation}
\begin{verbatim}
  DScalc:= DS[calc 1 as "temp_id" role "identifier"]
  DScond:= DScalc[filter business_id=100 and turnover>1000000]
  DSr:= DScond[keep (temp_id,business_id)][aggregate count (business_id)] = 1
\end{verbatim}
\subsubsection*{The eStatistik implementation}
\begin{verbatim}
  VAR rueck, hf_turnover, hf_undecided, hf_count
  rueck,hf_undecided, hf_count := {1,0,0}
  hf_turnover := LEER

  FUER JEDES hf_turnover IN MATERIAL mat_Rule11 (business_id = '100' ; turnover )

  WENN hf_turnover = LEER
  DANN hf_undecided := 1

  SONST
  WENN hf_turnover > 1000000
  DANN hf_count := hf_count + 1
  ENDE
  ENDE
  ENDE

  WENN hf_undecided = 1 ODER hf_count = 1
  DANN rueck := 0
  ENDE



  RUECKGABE rueck
\end{verbatim}
\subsubsection*{The validate implementation.}
\begin{verbatim}
  # rule_11:
  sum(business_id == 100 & turnover > 1e+06) == 1
\end{verbatim}


\newpage

\subsubsection*{  Rule 12: Natural language}
\begin{quote}


\begin{verbatim}
forall x: 
  IF x.relation_to_head = 4 
  THEN exists y:
    x.spouse-id = y.person-id AND y.relation_to_head = 3
\end{verbatim}

We assume all data is available.


\end{quote}
\subsubsection*{The VTL implementation}
\begin{verbatim}
  DS=person-id(identifier), spouse-id, relation_to_head

  DSfilter := DS[filter relation_to_head = 4]
  DSmerge := merge(DS "DSx",DS "DSy",
  on
  (DSy#spouse-id = DSx#person-id and DSy#relation_to_head = 3 and
DSx#relation_to_head = 4)
  return
  (DSx#person-id as "person-id"))

  DSnot_exists := DSfilter not_exists_in DSmerge

  DScount := DSnot_exists[calc 1 as "id" role "identifier"][keep (id, person_id)]
[aggregate count (person_id)] = 0
\end{verbatim}
\subsubsection*{The eStatistik implementation}
\begin{verbatim}
  VAR rueck,hf_relation_to_head
  rueck := {0}
  hf_relation_to_head := LEER

  WENN relation_to_head ='4'
  DANN

  WENN NICHT spouse_id IN MATERIAL mat_Rule12 (person_id)
  DANN rueck := 1
  SONST hf_relation_to_head := MATERIAL mat_Rule12 (person_id = spouse_id ;
relation_to_head )

  WENN hf_relation_to_head /= '3'
  DANN rueck := 1
  ENDE
  ENDE

  ENDE
  RUECKGABE rueck
\end{verbatim}
\subsubsection*{The validate implementation.}
\begin{verbatim}
  # def_rel_4:
  rel_4 := person_id[relation_to_head == 4]

  # def_rel_3:
  spouse_of_rel_3 := spouse[relation_to_head == 3]

  # rule_12:
  all(rel_4 %in% spouse_of_rel_3)
\end{verbatim}


\newpage

\subsubsection*{  Rule 13: Natural language}
\begin{quote}


The combination of sex and age group in the data set is unique, i.e., there do not exist two distinct records in
the data set with an identical combination of values for sex and age group.

- sex groups: \code{male}, \code{female}
- age groups: \code{child}, \code{adult}, \code{senior} 

We assume all data is available.


\end{quote}
\subsubsection*{The VTL implementation}
\begin{verbatim}
  DS= person-id(identifier),gender(identifier),age-group(identifier)
  /*
  * gender: male, female
  * age-groups: child, adult, senior
  */
  DScalc := DS[calc 1 as "id" role "identifier", 1 as "msrcount" role "measure"]
  DScount := DS[keep(id, msrcount, gender, age_groups)][aggregate count(msrcount)]
[filter msrcount > 1]
  DSr := DScount [keep (id, msrcount)][aggregate count(msrcount)] = 0
\end{verbatim}
\subsubsection*{The eStatistik implementation}
\begin{verbatim}
  VAR rueck, dummy, counter
  rueck, counter := {0,0}

  FUER JEDES dummy IN MATERIAL mat_Rule13 (gender = gender, age_group =
age_group ; person_id )
  counter := counter + 1


  WENN counter /= 1
  DANN rueck := 1
  ENDE

  ENDE
  RUECKGABE rueck
\end{verbatim}
\subsubsection*{The validate implementation.}
\begin{verbatim}
  # def_counts:
  counts := table(gender, age_group)

  # rule_13:
  counts <= 1
\end{verbatim}


\newpage

\subsubsection*{  Rule 14: Natural language}
\begin{quote}


Every combination of sex and age group occurs at least once in the data set.

- sex groups: \code{male}, \code{female}
- age groups: \code{child}, \code{adult}, \code{senior} 

We assume all data is available.


\end{quote}
\subsubsection*{The VTL implementation}
\begin{verbatim}
  DS=person-id(identifier), name, gender(identifier), age-group(identifier)
  DSgender= gender(identifier) {male, female}
  DSage =age-group(identifier) {child, adult, senior}
  /*
  * gender: male, female
  * age-groups: child, adult, senior
  */
  DSmerge := merge(DSgender "DSgender" ,DSage "DSage" ,
  on
  (1 = 1)
  return
  (DSgender#gender as "gender",DSage #age-group as "age-group"))
  DSdiff := DSmerge setdiff DS[keep (gender, age-group)]
  DSr := DSdiff [calc 1 as "msrcount" role "measure"][aggregate count(msrcount)] =
0
\end{verbatim}
\subsubsection*{The eStatistik implementation}
\begin{verbatim}
  VAR rueck, hf_age_group, hf_gender,
  male_child, female_child, male_adult, female_adult, male_senior, female_senior
  rueck,male_child,
female_child,male_adult,female_adult,male_senior,female_senior :=
{0,0,0,0,0,0,0}

  FUER JEDES hf_gender , hf_age_group IN MATERIAL mat_Rule14 (gender , age_group)
  WENN hf_gender = 'male' UND hf_age_group = 'child' DANN male_child := male_child
+ 1 ENDE
  WENN hf_gender = 'female' UND hf_age_group = 'child' DANN female_child :=
female_child + 1 ENDE
  WENN hf_gender = 'male' UND hf_age_group = 'adult' DANN male_adult := male_adult
+ 1 ENDE
  WENN hf_gender = 'female' UND hf_age_group = 'adult' DANN female_adult :=
female_adult + 1 ENDE
  WENN hf_gender = 'male' UND hf_age_group = 'senior' DANN male_senior :=
male_senior + 1 ENDE
  WENN hf_gender = 'female' UND hf_age_group = 'senior' DANN female_senior :=
female_senior + 1 ENDE

  ENDE
  WENN male_child = 0 ODER female_child = 0 ODER male_adult = 0 ODER female_adult
= 0 ODER male_senior = 0 ODER female_senior = 0
  DANN rueck := 1
  ENDE


  RUECKGABE rueck
\end{verbatim}
\subsubsection*{The validate implementation.}
\begin{verbatim}
  # def_counts:
  counts := table(gender, age_group)

  # rule_gender:
  gender %in% c("male", "female")

  # rule_ag:
  age_group %in% c("child", "adult", "senior")

  # rule_complete:
  counts == 1 && sum(counts) == 6
\end{verbatim}


\newpage

\subsubsection*{  Rule 15: Natural language}
\begin{quote}


If two records have the same postal code, they must have the same value for \code{city}. Below, this is expressed
as a functional dependency (https://en.wikipedia.org/wiki/Functional\_dependency)

\begin{verbatim}
postal_code --> city
\end{verbatim}



\end{quote}
\subsubsection*{The VTL implementation}
\begin{verbatim}
  DS=district-id(identifier),postcode(identifier),city(identifier)


  DScount := DS[calc 1 as msr_count role "MEASURE"]
  DSr := DScount[keep (postcode, msr_count)][aggregate count (msr_count)] =
  DScount[keep (postcode, city, msr_count)][aggregate count (msr_count)][keep
(postcode, msr_count)]
\end{verbatim}
\subsubsection*{The eStatistik implementation}
\begin{verbatim}
  VAR rueck, hf_city, counter
  rueck, counter := {0,0}

  FUER JEDES hf_city IN MATERIAL mat_Rule15 (postcode = postcode ; city )
  WENN city /= hf_city
  DANN rueck := 1
  ENDE

  ENDE
  RUECKGABE rueck
\end{verbatim}
\subsubsection*{The validate implementation.}
\begin{verbatim}
  # rule_15:
  postcode ~ city
\end{verbatim}


\newpage

\subsubsection*{  Rule 16: Natural language}
\begin{quote}


The following is a check on hierarchical aggreggation.

\begin{verbatim}
forall k >= 1: w(x1. ... .xk) equals the sum of
w(x1. ... .xk.i) forall i >= 0
\end{verbatim}

We assume all data is available.



\end{quote}
\subsubsection*{The VTL implementation}
\begin{verbatim}
  DS=id(identifier),level(identifier),weight

  /*
  * Create a hierarchy (actually is no possible to do using VTL because some
string operators are missing)
  *
  * MAPS FROM MAPS TO LEVEL SIGN
  * x1 1 +
  * x1.1 x1 2 +
  * x1.2 x1 2 +
  * x1.3 x1 2 +
  * x2 1 +
  * x2.1 x2 2 +
  */

  DShierarchy := hierarchy(DS, level, "HRC", false)
  DScond := (DShierarchy = DS)[filter weight = "false"]
  DSr := DScond[calc 1 as "msrcount" role "MEASURE"][aggregate count(msrcount)] =
0
\end{verbatim}
\subsubsection*{The eStatistik implementation}
\begin{verbatim}
  VAR rueck, hf_AnzSst, hf_such, hf_level, hf_sum, hf_weight, hit
  rueck,hf_sum,hit := {0,0,0}

  hf_AnzSst := FUNKTION WERTLAENGE (level)

  WENN hf_AnzSst IN REIHE (1,3)
  DANN

  FUER JEDES hf_level, hf_weight IN MATERIAL mat_Rule16 (level, weight )

  WENN hf_AnzSst = 1 UND FUNKTION WERTLAENGE (hf_level) = 3 UND
  FUNKTION TEIL (hf_level,1,1) = FUNKTION TEIL (level,1,1)
  DANN hf_sum:= hf_sum + hf_weight
  hit := 1
  ENDE

  WENN hf_AnzSst = 3 UND FUNKTION WERTLAENGE (hf_level) = 5 UND
  FUNKTION TEIL (hf_level,1,3) = FUNKTION TEIL (level,1,3)
  DANN hf_sum:= hf_sum + hf_weight
  hit := 1
  ENDE

  ENDE

  "Fehlerermittlung"
  WENN hf_sum /= weight UND hit = 1
  DANN rueck := 1
  ENDE

  ENDE

  RUECKGABE rueck
\end{verbatim}
\subsubsection*{The validate implementation.}
\begin{verbatim}
  # def_parent:
  with_parent := sub("\\.?\\d+$", "", level)

  # def_is_parent:
  is_parent := level %in% with_parent

  # def_par_lev:
  parent_levels := level[is_parent]

  # def_childsum:
  childsum := tapply(weight, with_parent, sum)

  # rule_16:
  weight[is_parent] == childsum[parent_levels]
\end{verbatim}


\newpage

\subsubsection*{  Rule 17: Natural language}
\begin{quote}


The value for no\_of\_household\_members must equal the number of records for each household


\end{quote}
\subsubsection*{The VTL implementation}
\begin{verbatim}
  DShousehold=household-id(identifier),members
  DSpersons=person-id(identifier),household-id(identifier) (in the example fields
are not correctly defined)

  DScount := (DSpersons[calc 1 as "members" role "MEASURE"][keep (household-id,
members)][aggregate count(members)]=
  DShousehold)[filter members= "false"]
  DSr := DScount[calc 1 as "msr_count" role "MEASURE"][aggregate count(msr_count)]
= 0

\end{verbatim}
\subsubsection*{The eStatistik implementation}
\begin{verbatim}
  VAR rueck, hf_dummy, counter
  rueck, counter := {0,0}

  FUER JEDES hf_dummy IN MATERIAL personen (household_id = household_id ;
person_id )
  counter := counter + 1
  ENDE

  WENN counter /= members
  DANN rueck := 1
  ENDE

  RUECKGABE rueck
\end{verbatim}
\subsubsection*{The validate implementation.}
\begin{verbatim}
  # def_count:
  person_count := table(person$household_id)

  # rule17:
  members == person_count[household_id]
\end{verbatim}


\newpage

\subsubsection*{  Rule 18: Natural language}
\begin{quote}


This last one is a bit complicated. It involves two files, one with households (\code{x}) and one with persons data (\code{y}). In the household file, it is registered how many members there are, say 3. It is then expected that
there are persons with \code{person-id} 1,2,3 in the file \code{y}. The rule is satisfied if for all households, all person-id's can be found, and the id's have the correct values.


\begin{verbatim}
forall x: forall n:
  IF 1 <= n <= x.no\emph{of}household\_members
  THEN exists y: 
    x.household-id = y.household-id AND y.person-id = n
\end{verbatim}


We assume all data is available.

\end{quote}
\subsubsection*{The VTL implementation}
\begin{verbatim}
  DShousehold=household-id(identifier),members
  DSpersons=household-id(identifier), person-id(identifier)


  DSmerge:=merge (DShousehold as "DSh", DSpersons as "DSp"
  on DSh#household-id=DSp#household-id,
  return
  (DSh#household-id as household-id,DSh#person-id as person-id,DSp#members as
members))


  DSout:= DSmerge[filter person-id < 1 or person-id>members][keep (household-
id,members)][aggregate count (members)] = 0

  DSdist:= DSmerge[rename (person-id) as "p_id" role "measure"][aggregate
count_distinct (p_id)][filter p id <> members]
  [aggregate count (members)] = 0


  DSr := (DSout and DSdist)

\end{verbatim}
\subsubsection*{The eStatistik implementation}
\begin{verbatim}
  VAR rueck, hf_person_id, x1
  rueck := 0

  WIEDERHOLE FUER x1 := 1 BIS x1 > members
  WENN NICHT REIHE (household_id, x1) IN MATERIAL personen (household_id ,
person_id )
  DANN rueck := 1
  ENDE

  ENDE
  "Anmerkung: "
  "Fall Anz. der Personensätze > der HH-members wird in Rule_11 abgefangen"
  "-> Daher keine Überprüfung mehr"

  RUECKGABE rueck
\end{verbatim}
\subsubsection*{The validate implementation.}
\begin{verbatim}
  # def_max:
  total_members := household$members[match(household_id, household$household_id)]

  # rule18:
  person_id >= 1 && person_id <= total_members
\end{verbatim}

