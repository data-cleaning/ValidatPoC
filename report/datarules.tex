\title{Data and Rules}
\label{datarules}

The rule set for this proof of concept was chosen to on one hand reflect rules
that are used in practice and on the other hand cover a wide range of rule
types.  For this reason, two rules sets were combined. 

The first rule set was obtained by first classifying the rules obtained through
the ESS survey \citep{walsdorfer:2015}, and picking a rule from each formal
class defined in the handbook \citep{loo:2015c, zio:2015}. This formal
classification divides validation rules into ten disjunct classes,
which can be aggregated in five levels of complexity. 
The idea of this typology is that each data value
is identified by a set of four not completely independent keys that label:
\begin{itemize}
\item The domain: $U$,
\item the time of measurement $\tau$. This determines the population,
\item the selected statistical unit: $u$,
\item the variable that is measured: $X$.
\end{itemize}
Validation functions are labeled according to whether they are a function of a
single ($s$) or multiple $(m)$ instances of these keys. For example, the rule
$x > 0$ is labelled $ssss$ and the rule $x + y = z$ is labelled $sssm$  (since
it involves multiple variables. Since the keys are not completely independent
there are 10, not 16 independent categories.  As it turned out, nine of the ten
classes (and four of the five complexity levels) actually occur in rules
reported to the survey.

The second rule set was obtained by taking nine rules constructed for the study
of VTL by \cite{gelsema:2015}. These rules were constructed to test the limits
of what can be expressed by a language, and not necessarily for their
usefulness. The difference between the rules is mainly based on what operators
from first order logic are necessary to describe them.


\begin{table}[p]
\begin{tabular}{|l|lllp{0.7\textwidth}|}
\hline
&\multicolumn{4}{l|}{\textbf{Rules from the survey}}\\
\hline
\textbf{No} & \textbf{Level} & \textbf{Class} & \textbf{Domain} & \textbf{Rule} \\
\hline
1 & 0 & $ssss$ & LFS & Number of hours per week usually worked should be between 1 and 80\\
\hline
2 & 1 & $sssm$ & SBS & Cost + Profit = Turnover\\
\hline
3 & 1 & $ssms$ & Census & Check whether the relative occurrence of the category
high in a column containing values low, high, medium does not exceed 10\%.\\
\hline
4 & 1 & $smss$ & Prices  &price change comparing to previous month and
corresponding month of the previous year does not exceed 50\%\\
5 & 2 & $ssmm$ & Census & Age of grandparents – 28 $\geq$ age of their grandchildren\\
\hline
6 & 2 & $smsm$ & Agriculture & If a product is out of season, the price and
quantity must be the same as last month's values.\\
\hline
7 & 2 & $smms$ & Prices & The price change of a single item may not influence
the change in the mean prices by more than 10\%, upwards or downwards.\\
\hline
8 & 3 & $msmm$ & Census & Year of birth in household questionnaire must equal
year of birth in individual questionnaire\\
\hline
&\multicolumn{4}{l|}{\textbf{Rules from the Study of VTL}}\\
\hline
9  & \multicolumn{4}{p{1.1\textwidth}|}{\code{forall x: x.age >= 0 AND x.age <=
113}}\\
\hline
10 & \multicolumn{4}{p{1.1\textwidth}|}{\code{exists x: x.business-id = 100 AND
x.turnover $>$ 1.000.000}}\\
\hline
11 & \multicolumn{4}{p{1.1\textwidth}|}{\code{exists! x: x.business-id = 100 AND
x.turnover $>$ 1.000.000}}\\
\hline
12 & \multicolumn{4}{p{1.1\textwidth}|}{\code{forall x: IF x.relation\_to\_head
= 4 THEN exists y: x.spouse-id = y.person-id AND y.relation\_to\_head = 3}}\\
\hline
13 & \multicolumn{4}{p{1.1\textwidth}|}{The combination of sex and age group in
the data set is unique, i.e., there do not exist two distinct records in the
data set with an identical combination of values for sex and age group.}\\
\hline
14 & \multicolumn{4}{p{1.1\textwidth}|}{Every combination of sex and age group
occurs at least once in the data set.}\\
\hline
15 & \multicolumn{4}{p{1.1\textwidth}|}{If two records have the same postal
code, they must have the same value for city}\\
\hline
16 & \multicolumn{4}{p{1.1\textwidth}|}{\code{forall k >= 1: w(x1. ... .xk)
equals the sum of w(x1. ... .xk.i) forall i $>=$ 0}}\\
\hline
17 & \multicolumn{4}{p{1.1\textwidth}|}{The value for no\_of\_household\_members
must equal the number of records for each household}\\
\hline
18 & \multicolumn{4}{p{1.1\textwidth}|}{
\code{forall x: forall n:
  IF 1 $<=$ n $<=$ x.no\_of\_household\_members
  THEN exists y: 
    x.household-id $=$ y.household-id AND y.person-id $=$ n
}}\\
\hline
\end{tabular}
\caption{Overview of rules used in the proof of concept}
\label{tab:rules}
\end{table}

