\textsc{Mark van der Loo and Michael Schaefer}
\vspace{0.6 cm}

The evaluation of existing languages for specifying data validation rules is
one of the main objectives of the ESSnet on Validation project \citep{ESS:2015}. Specifically, two types of tasks are defined in work package 4,
namely the conduction of a formal study (task 15), described in [study] and the
implementation of a prototype, described in \textsc{this document}. Originally,
two candidate languages were considered for evaluation: VALS, which had already
been under development for some years under the auspices of Eurostat, and VTL,
a more recent development from the SDMX community, of which Eurostat is a
member. Given to circumstances not forseeable at the outset of the project, the
development of VALS was ceased. This is described in more detail in the
introduction to [study].

The objective of the prototype described in this document was to assess the
practical usability of VTL when used for specifying and exchanging validation
rules in the ESS. Given the current lack of editors and runtimes for VTL, it
was decided to spin the prototype more into the direction of a practical
approach to integrating VTL into existing data validation systems. Fortunately,
two of the project partners, the Dutch CBS and the German FSO, have developed
productive data validation systems -  \textit{Validate} and
\textit{eSTATISTIK}, respectively - that both feature full-fledged data
validation specification languages, but differ fundamentally in their technical
layouts and integrations with end-users and production systems. Including both
systems into the prototype suggested itself as it promised to increase the
relevance of the findings considering the diversity of environments in which
data validation is performed in the ESS.
